\section{Correlation energies from the Random Phase Approximation}
\subsection{The adiabatic connection}
\begin{exercise}
Show the Hellman-Feynman theorem.
\begin{equation}
    \dv{\lambda}\mel{\Psi_{\lambda}}{\hat{A}_{\lambda}}{\Psi_{\lambda}} = \mel{\Psi_{\lambda}}{\dv{\lambda}\hat{A}_{\lambda}}{\Psi_{\lambda}}
\end{equation}
(Hint: Use that $\braket{\Psi_{\lambda}}{\Psi_{\lambda}}=1$ for all $\lambda$) 
\end{exercise}

\begin{solution}
Upon applying the operator on both sides we obtain
\begin{equation}
        \dv{\lambda}\mel{\Psi_{\lambda}}{a_{\lambda}}{\Psi_{\lambda}} = \mel{\Psi_{\lambda}}{\dv{\lambda}a_{\lambda}}{\Psi_{\lambda}}
\end{equation}
$a_{\lambda}$ can be moved outside the integral, and so can the derivative in the second term as the integral is not over $\lambda$, thus
\begin{equation}
    \dv{\lambda} a_{\lambda} = \dv{\lambda} a_{\lambda}
\end{equation}
Where the hint was used, thus it is shown.
\end{solution}